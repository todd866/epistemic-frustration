\documentclass[11pt]{letter}
\usepackage[margin=1in]{geometry}
\usepackage{hyperref}

\signature{Ian Todd\\Sydney Medical School\\University of Sydney}
\address{Ian Todd\\Sydney Medical School\\University of Sydney\\Sydney, NSW, Australia\\itod2305@uni.sydney.edu.au}

\begin{document}

\begin{letter}{Editorial Office\\Biology \& Philosophy}

\opening{Dear Editors,}

I am pleased to submit ``Epistemic Frustration: Dimensional Collapse and the Limits of Falsifiability in Complex Systems'' for consideration at \textit{Biology \& Philosophy}.

\textbf{Summary.} The paper develops a framework for understanding how complex adaptive systems---including biological organisms, scientific paradigms, and social institutions---transition from exploration-phase epistemics (truth-seeking) to exploitation-phase epistemics (coordination-maintenance). I introduce \emph{epistemic frustration}: the condition where high-dimensional optima project into mutually contradictory positions in low-dimensional discourse, making some disagreements geometric rather than epistemic. I propose an operational diagnostic for unfalsifiability based on degrees-of-freedom counting.

\textbf{Why Biology \& Philosophy.} The paper's central case study concerns the free will debate, reframed as a question for philosophy of biology rather than philosophy of mind. I argue that the key question is not ``do humans have free will despite being physical systems?'' but ``is agency a fundamental feature of life that we have been systematically misdescribing as mechanism?''

The case study traces the implications: if agency is widespread in living systems, where do we draw boundaries? Does a bacterium have agency? A cell? I propose that agency scales with the gap between internal complexity and external measurability---making it a matter of degree rather than kind. This reframes debates about biological mechanism, the limits of reductionism, and the persistent difficulty of reducing biology to physics.

Importantly, the paper does not claim to resolve the free will debate metaphysically. It argues that \emph{regardless of the metaphysics}, the sociology of the debate follows coordination-first dynamics: phenomenological evidence is dismissed, dissent signals naivety, and the paradigm has become unfalsifiable by its own criteria. This is a meta-level claim about how the debate is conducted, not a first-order claim about who is correct.

\textbf{Contribution.} The paper offers:
\begin{enumerate}
\item A formal concept (\emph{epistemic frustration}) for projection-induced incompatibility
\item An operational diagnostic for when paradigms enter unfalsifiable regimes
\item A sustained case study on agency and free will, framed for philosophy of biology
\item A case study on biological senescence as ``epistemic frustration at the cellular level''---the organism losing its capacity to explore novel repair strategies
\item Additional case studies demonstrating cross-domain recurrence under shared scope conditions
\end{enumerate}

The manuscript is approximately 48 pages and includes 5 figures.

Thank you for considering this submission. I look forward to hearing from you.

\closing{Sincerely,}

\end{letter}
\end{document}
